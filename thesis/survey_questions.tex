\section{Fragen der Umfrage}
\label{sec:survey_questions}
\begin{enumerate}
    \item Frage "Wie würdest du deine Fähigkeit (nach dem Spielen) einschätzen, Deepfake-Videos zu erkennen?" mit den Antwortmöglichkeiten als Likert Skala von 1-5
    \item Frage "Welche Erfahrungen hast du mit Deepfakes bereits gemacht?" (multiple-choice, nur bei Antwort d single-choice)
    \begin{enumerate} [label=\alph*)]
        \item Ich habe bereits Deepfakes gesehen (z. B. in sozialen Medien, Nachrichten)
        \item Ich habe bereits Deepfakes erstellt
        \item Ich habe bereits Spiele zur Erkennung von Deepfakes gespielt
        \item Ich habe noch keine Erfahrung mit Deepfakes
        \item Andere: (offene Texteingabe)
    \end{enumerate}
    \item Frage "Ich habe durch das Spiel neue Informationen über die Erkennung von Deepfakes gewonnen.": Likert Skala von 1-7
    \item Frage "Hast du die vorgestellten Strategien zur Erkennung von Deepfakes gelesen?" (single-choice)
    \begin{enumerate} [label=\alph*)]
        \item Ja 
        \item Nein 
    \end{enumerate}
    \item Wenn der Nutzer mit "Ja" geantwortet hat: 
        Frage "Die Strategien waren für mich neu.": Likert Skala von 1-7
    \item Wenn der Nutzer mit "Nein" geantwortet hat:
	    Frage "Warum hast du die vorgestellten Strategien nicht gelesen?" (multiple-choice)
    \begin{enumerate} [label=\alph*)]
        \item Ich hatte nicht genug Zeit
        \item Ich dachte, ich bräuchte keine zusätzliche Hilfe 
        \item Ich habe sie übersehen/ nicht bemerkt 
        \item Ich wollte das Spiel zuerst selbst ausprobieren 
        \item Sonstiges: (offene Texteingabe)
    \end{enumerate}
    \item Frage "Hast du das Spiel mehrmals gespielt?" (single-choice)
    \begin{enumerate} [label=\alph*)]
        \item Ja 
        \item Nein 
    \end{enumerate}
    \item Wenn der Nutzer bei dieser Frage und der vorherigen Frage, 
        ob er die Strategien gelesen hat, mit "Ja" geantwortet hat: 
        Frage "Die Strategien, die ich bei meinen früheren Durchgängen gelernt habe, 
        haben meine Fähigkeit verbessert, Deepfakes in den folgenden Spielsitzungen zu erkennen.":
		Likert Skala von 1-7 
    \item Frage "Hast du schon einmal ein Spiel auf einem Public Display gespielt?" (single-choice)
    \begin{enumerate} [label=\alph*)]
        \item Ja 
        \item Nein 
        \item Weiß ich nicht 
    \end{enumerate}
    \item Frage "Die spielerische Form hat meine Motivation zum Lernen der Deepfake Erkennung erhöht.":
	    Likert Skala von 1-7
    \item Frage "Welche Funktionen hast du besonders motivierend gefunden?" (multiple-choice)
    \begin{enumerate} [label=\alph*)]
        \item Spiel speichern 
        \item Strategien 
        \item Statistiken 
        \item Videos von prominenten Personen 
        \item andere Funktionen: (offene Texteingabe)
    \end{enumerate}
    \item Frage "Was hat dich auf das Spiel aufmerksam gemacht?" (multiple-choice)
    \begin{enumerate} [label=\alph*)]
        \item Interesse an Deepfakes
        \item Die auffällige Animation (Papst/Trump)
        \item Die spielerische Form
        \item Ich habe auf Jemanden gewartet
        \item Ich hatte Zeit über
        \item Sonstige Gründe: (offene Texteingabe)
    \end{enumerate}
    \item Frage "Welche Erfahrungen hast du mit Public Displays bereits gemacht?" 
        (multiple-choice, nur bei Antwort c single-choice)
    \begin{enumerate}
        \item Ich habe bereits Public Displays gesehen (in öffentlichen Verkehrsmitteln, Einkaufszentren usw.)
        \item Ich habe bereits mit Public Displays interagiert (als Spiel, zu Informationszwecken usw.)
        \item Ich habe noch keine Erfahrung mit Public Displays
	    \item Sonstiges: (offene Texteingabe)
    \end{enumerate}
    \item Frage "Warum hast du aufgehört, das Spiel zu spielen?" (multiple-choice)
    \begin{enumerate}
        \item Ich habe das Interesse an dem Spiel verloren
        \item Mir stand nur begrenzt Zeit zur Verfügung
        \item Ich habe alle Videos angesehen
        \item Ich kann keinen besonderen Grund nennen
        \item andere Gründe: (offene Texteingabe)
    \end{enumerate}
    \item Frage "Wie alt bist du?": Alter 
    \item Frage "Welchem Geschlecht gehörst du an?" (single-choice)
    \begin{enumerate}
        \item weiblich 
        \item männlich 
        \item divers 
        \item anders 
    \end{enumerate}
    \item Frage "Welches ist dein höchster Abschluss?" (single-choice)
    \begin{enumerate}
        \item Abitur oder niedriger
        \item Bachelor oder vergleichbar
        \item Master oder vergleichbar
        \item PhD
        \item Anderes: (offene Texteingabe)
    \end{enumerate}
    \item Frage "Wie ist dein derzeitiger Beschäftigungsstatus?"
    \begin{enumerate}
        \item Student
        \item Erwerbstätig
        \item Selbstständig
        \item Arbeitslos
        \item Rentner
        \item Anderes: (offene Texteingabe)
    \end{enumerate}
    \item Frage "Was ist dein Fachgebiet?"
    \begin{enumerate}
        \item Informatik
        \item Gesundheitswissenschaften
        \item Mathematik/Physik
        \item Psychologie
        \item Anderes: (offene Texteingabe)
    \end{enumerate}
    \item Frage "Hast du noch zusätzliches Feedback oder Vorschläge zum Spiel, die du uns mitteilen möchtest?" (offene Texteingabe)	
\end{enumerate}
Anmerkung: Bei der Antwortmöglichkeit "Likert Skala von 1-7" hat der Nutzer im Detail folgende Antwortmöglichkeiten: 
Stimme überhaupt nicht zu - Stimme nicht zu - Stimmer eher nicht zu - Stimme weder zu noch nicht zu - Stimme eher zu - Stimme zu - Stimme völlig zu
